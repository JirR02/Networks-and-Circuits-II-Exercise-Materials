\documentclass[11pt,a4paper]{article}
\usepackage{tipa}
\usepackage{emoji}

\usepackage[utf8]{inputenc} % Zeichenkodierung
\usepackage[T1]{fontenc}    % Bessere Zeichenunterstützung
\usepackage[german]{babel}  % Korrekte deutsche Typografie
\usepackage[a4paper, left=2cm, right=2cm, top=2.5cm, bottom=2.5cm]{geometry}
\usepackage{amssymb, amsmath, amsthm}
\usepackage{tikz}
\usepackage{graphicx}
\usepackage[export]{adjustbox}
\usepackage{microtype}
\usepackage{epstopdf}
\usepackage{float}
\usepackage{amsmath}
\usepackage{amssymb}
\usepackage{amstext}
\usepackage{amsfonts}
\usepackage{mathrsfs}
\usepackage{hyphenat}
\usepackage{fancyhdr}  
\usepackage{qrcode}  % QR-Code generieren
\setlength{\skip\footins}{40pt}
\title{}
\date{}  % Kein Datum auf Standard-Titel

\begin{document}

% === DECKBLATT ===
\begin{titlepage}
    \centering
    {\Huge Netzwerke und Schaltungen II}\\[0.8cm]
    {\Large D-ITET}\\[0.8cm]
    {\Large HS2025}\\[3.5cm]
    
    {\Huge \textbf{Übung X}}\\[1cm]
    {\Large 1.1.2000}\\[3.5cm]

    \qrcode[height=6cm]{https://www.example.com}

    \vfill
    {\Large \textbf{Rares Sahleanu}}
\end{titlepage}

% === INHALTSVERZEICHNIS ===
%\newpage
%\tableofcontents

% === AKTIVIERT KOPF- UND FUSSZEILEN AB SEITE 3 ===
\newpage
\pagestyle{fancy}  

% Kopfzeile: Links - Mitte - Rechts (subtil & kleiner)
\fancyhead[L]{\small \textit{Netzwerke und Schaltungen II}}  
\fancyhead[C]{\small \textit{Übung X}}
\fancyhead[R]{\small \textit{Rares Sahleanu}}

% Fußzeile: Seitenzahl in der Mitte
\fancyfoot[L]{}  
\fancyfoot[C]{\thepage}
\fancyfoot[R]{}

% Linie in der Kopfzeile aktivieren, Fußzeile ohne Linie
\renewcommand{\headrulewidth}{0.4pt}  
\renewcommand{\footrulewidth}{0pt}  

% === START DES DOKUMENTS ===
\section{Organisatorisches}
Kurz zu mir: Ich heiße Rares\footnote{ausgesprochen \textit{Raresch} \textipa{['ra:r\esh]}}, bin 19 Jahre alt und sitze mein 4. Semester am ITET ab. Zu meinen größten Errungenschaften gehören der Rank \textit{Distinguished Master Guardian} in CS:GO, eine aufstrebende Hobby-Rapper\footnote{Man kennt mich unter einer Vielzahl von Künstlernamen: LilReez, LilReezy, RaresDerAgrarmensch, Reez} \textit{karriere und +6.00 CHF} Endbilanz bei swisslos.ch\footnote{Sportwetten sind nicht meins. Meistens Wettet der ``Zug-Typ`` Eddy und ich nehme die Rolle der Opposition ein, indem ich seine Entscheidung runterrede}. 

\subsection{FAQ}
Hier ist eine Liste der Fragen die oft gestellt werden:

\subsubsection{``Wie ist das Fach XY{?}``}
Im zweiten Semester liegt die Schwierigkeit weniger in der \textit{Komplexität} des Stoffes, sondern mehr in der Menge. Es ist VIEL STOFF, aber dafür ist er nicht allzu ``schwer`` zu verstehen. Ach und: Wenn man nicht Programmieren kann/bzw. keine Vorerfahrungen in Informatik hat, dann sollte man Informatik sehr sehr sehr ernst nehmen. Hier eine kleine Übersicht:

\begin{itemize}
    \item \textbf{Analysis 1/2}: Am Ball bleiben und Serien lösen! Besonders Ziltener hält sich sehr nahe an den Serien und seine Klausuren sind fair. Die Prüfung geht ``nur`` 4 Stunden und da kann er nicht alles abfragen. Es kommen nur Key-Conecpts dran, welche man gut üben kann
    \item \textbf{Komplexe Analysis}: Ein ``normaler`` Kurs. Es lässt sich alles gut grafisch Vorstellen und die Aufgaben in der Klausur sind sehr ``absehbar``. Unterschätzen sollte man es nicht, aber zu viel Zeit investieren auch nicht
    \item \textbf{Physik}: Inhaltlich hält es sich sehr nah an den anderen Fächern. Schwingungen sind basically eine Carbon-Copy von den anderen Fächern. Gut ``übbar`` und mit Notenbonus ist die Klausur machbar
    \item \textbf{Informatik}: Squid-Game in Reallife. Wenn man Programmieren kann , dann easy. Wenn nicht, dann ist es ein ernst zu nehmendes Fach, sonst wirklich Krise. Hier gilt: Üben, Üben, Üben
    \item \textbf{Netzwerke und Schaltungen}: Neben Analysis das wichtigste Fach. Der Stoff hält sich in Grenzen\footnote{Wie Geburtstagstortenkerzen - Fragt falls ihr das nicht verstanden habt} aber in der Klausur wird hauptsächlich Schnellligkeit und Routine abgefragt. Wer viel übt holt i.d.R. gute Noten

\end{itemize}

\subsubsection{``Was kannst du zum Üben empfehlen{?}``}
Für Informatik Empfehle ich ganz klar die C++ Bibel\footnote{978-3836298537} von THorsten Will in Kombination mit LeetCode. Für Analysis ist es zu 100\% der TA Angelo Nujic\footnote{https://polybox.ethz.ch/index.php/s/UxNajbQ3tLOh4pH} und die Übung. Für NuS lege ich euch die Probeprüfungen ans Herz. Die Prüfung ist immer die selbe, weshalb sich üben mit den Probeprüfungen lohnt!

\subsubsection{``Ich habe in Analysis 1 nicht ganz aufgepasst - Ist das schlimm{?}``}
Nope. Ich war selbst nur in den ersten beiden Vorlesungen von Analysis 1 und hab mir vereinzelt Aufzeichnungen angeschaut und ca. 30\% der Serien gemacht. In Analysis 2 wird eh alles neu aufgerollt

\subsubsection{``Ich bin in BP A gerado so durchgekommen. Ist das ein Problem``}
Entgegen aller Gerüchte ist das 2. Semester nicht komplizierter sondern nur schwer. Was zuerst paradox wirkt bedeutet, dass man nur viel zu tun hat, aber es gibt kein Stoff mehr ``den man etwas nicht mehr versteht, weil es kienen sinn macht``

\subsection{Overview Übungsstunde}
\centering

% Erster Eintrag
\begin{minipage}{0.8\linewidth}
    Website zu der Übungsgruppe
    \hfill  
    \qrcode[height=4cm]{https://n.ethz.ch/~rsahleanu/}
\end{minipage}

\vfill % Gleichmäßiger Abstand zwischen den Einträgen
\begin{minipage}{0.8\linewidth}  
    Whatsapp-Gruppe
    \hfill  
    \qrcode[height=4cm]{https://chat.whatsapp.com/JprzBOwpTco32ea71sphA3}
\end{minipage}
\vfill
\begin{minipage}{0.8\linewidth}
   
    Umfrage zum Format
    \hfill  
    \qrcode[height=4cm]{https://docs.google.com/forms/d/e/1FAIpQLSf3egaMMpo945ZH_U_chb6AnNVzI2xlbR-GqT1HiwygXTD-CQ/viewform?usp=header}
\end{minipage}
\newpage
\raggedright
% === ZWEITES KAPITEL ===
\section{Effektiv- und Scheitelwert}
Soooo fangen wir erstmal mit den Basics an :) Bevor es so richtig los geht müssen wir aber erstmal ein paar Definition machen.

\subsection{Scheitelwert/Spitzenwert}
Der Scheitelwert ist einfach der maximale Wert den eine Schwingung erreicht

\subsection{Periodendauert/Frequenz/Winkelfrequenz}
Die Periodendauer einer Frequenz gibt an ``wie lange`` es dauert bis die Schwingung sich Wiederholt. Die Frequenz ist dementsprechend $\frac{1}{T}$ und gibt an wie oft pro Sekunde sich eine Schwingung "wiederholt". 
$\rightarrow$ Die \textit{Winkelgeschwindigkeit} $\omega=\frac{2\pi}{T} = 2\pi f$ ``mappt`` das Signal zuerst auf einem Kreis und gibt an ``wie schnell`` man sich drehen muss um das Signal korrekt zu lesen


\subsection{Mittelwert/Gleichrichtwert/Effektivwert}
Um Wechselspannung zu beschreiben benötigen wir einige \textit{represäntativen} Werte, die aber Gott sei Dank alle inutuitiv sind :)

\begin{itemize}
    \item\textbf{Mittelwert}\footnote{In der Klausur ist sowas i.d.R. immer gleich 0. Man kann hier oft Symmetrien ausnutzen} \quad $\bar{u} = \frac{1}{T} \int_{t=t_0}^{t=t_0+T} u(t) \, dt \rightarrow$ Fläche unter dem Graphen über eine Schwingung
    
    \item\textbf{Gleichrichtwert} \quad $|\bar{u}| = \frac{1}{T} \int_{t=t_0}^{t=t_0+T} |u(t)| \, dt\rightarrow$  Die Fläche unter dem Graphen wenn man die negativen Anteile nach ``oben klappt``
    
    \item\textbf{Effektivwert}\footnote{bei $\sin$ oder $\cos$ kann man Symmetrien ausnutzen} \quad $U = \sqrt{\frac{1}{T} \int_{t=t_0}^{t=t_0+T} u(t)^2 \, dt}\rightarrow$ Gibt an welche Gleichspannung die selbe Leistung liefern würde
\end{itemize}

\subsection{Zusatzaufgaben}
\begin{itemize}
    \item Aufgabe 1: Bestimmen Sie die Grenzfrequenz eines einfachen RC-Tiefpassfilters.
    \item Aufgabe 2: Zeichnen Sie das Bode-Diagramm eines gegebenen RLC-Bandpassfilters.
\end{itemize}

\section{Zeigerdarstellung}
Wir werden in NuS II hauptsächlich mit Zeigern arbeiten. Diese dienen als Brücke zwischen unserer echten Welt mit realen Zahlen und den komplexen Zahlen. Obwohl es ggf. zu Beginn nicht so schneint erleichtern uns letztere das Leben :)

\subsection{Unser tägliches Brot}
Jede mathematische Entdeckung wird immer nach deren zweiten Entdecker benannt. Wieso? Weil der erste immer Leonhard  Euler\footnote{Neben einem funktionierenden Bahn-System und ``El Tony``-Mate wahrscheinlich das beste was die Schweiz jemals hervorgebracht hat} war. Nachdem ist unter anderem folgendene Formel beannt.






\theoremstyle{definition}
\newtheorem*{definition}{Satz: Euler Formel}
\begin{definition}
Sei $\varphi$ ein Winkel so gilt:
\[
\cos\{\varphi\} + j \sin\{\varphi\} = e^{j\varphi}
\]
Wobei j die ``imaginäre Einheit`` ist. Sie ergibt sich aus $j = \sqrt{-1}$
\end{definition}

\noindent Das können wir nun mithilfe von ein paar Konventionen für unsere Schwingenden Signale nutzen.

\subsection{Von der Schwingung zum Zeiger}

Nehmen wir nun an, dass eine Schwingung sinusförmig ist. So lässt sie sich schreiben als:

\[
u(t) = \hat{u}\cos(\omega t + \varphi) = Re\{\hat{u}e^{j\omega t + j\varphi}\} = Re\{\hat{\underline{u}} e^{j\omega t}\} 
\] mit $\hat{\underline{u}} = \hat{u} e^{j\varphi}$, welcher der sog. ``Zeiger``\footnote{Der Drehanteil $ e^{j\omega t}$ ist für uns meist nicht von Bedeutung. Das wird aber während der Vorlesung klarer} ist.
\vspace{0.3cm}
\newline
Das kann am Anfang etwas verwirrend sein. Versucht es euch vorzustellen, wie ein rotierender Pfeil, welcher von Oben mit Licht bescheint wird. Um die Schwingung zu kriegen müssen wir nur den Schatten ablesen

\subsection{Nutzen}
Zeiger bilden einen eigenen Raum. Das heißt wird können Schwingungen eindeutige Zeiger zuweisen und umgekehrt. Ebenfalls gilt in diesem Raum das Distributiv, Assozitativ, Kommutativgesetz. Kurs: Um zwei Schwingungen zu Addieren können wir einfach deren Zeiger Addieren und dann zurück "wandeln". Hier eine Übersicht:


\begin{itemize}
    \item \textbf{Addition von Schwingungen:} \quad 
    \[
    a(t) + b(t) \quad \rightarrow \quad (\hat{\underline{a}} + \hat{\underline{b}}) e^{j\omega t}
    \]

    \item \textbf{Multiplikation mit einer Konstanten:} \quad 
    \[
    c \cdot a(t) \quad \rightarrow \quad (c\cdot \hat{\underline{a}}) e^{j\omega t}
    \]

    \item \textbf{Differenzbildung:} \quad 
    \[
    a(t) - b(t) \quad \rightarrow \quad (\hat{\underline{a}} - \hat{\underline{b}}) e^{j\omega t}
    \]

    \item \textbf{Skalierung von Amplituden:} \quad 
    \[
    k \cdot a(t) \quad \rightarrow \quad (k\cdot \hat{\underline{a}}) e^{j\omega t}
    \]

    \item \textbf{Phasenverschiebung:} \quad 
    \[
    a(t + \tau) \quad \rightarrow \quad \hat{\underline{a}}\cdot e^{j\omega \tau}
    \]


% Linie über die gesamte Textbreite mit rechtsbündigem Text
\noindent
\begin{tikzpicture}
    \draw[dashed] (0,0) -- (\linewidth,0) 
        node[anchor=east, above, xshift=-1.4cm, opacity=0.5] {\small Bis hierhin wichtig}
        node[anchor=east, below, xshift=-1.4cm, opacity=0.5] {\small Ab hier unwichtig};
\end{tikzpicture}



    \item \textbf{Modulation (Multiplikation mit einer anderen Schwingung):} \quad 
    \[
    a(t) \cdot \cos(\Omega t) \quad \rightarrow \quad \frac{\hat{\underline{a}}}{2} \left( e^{j(\omega+\Omega)t} + e^{j(\omega-\Omega)t} \right)
    \]

    \item \textbf{Differentiation im Zeitbereich:} \quad 
    \[
    \frac{d}{dt} a(t) \quad \rightarrow \quad j\omega\cdot \hat{\underline{a}} \cdot e^{j\omega t}
    \]
 
    \item \textbf{Integration im Zeitbereich:} \quad 
    \[
    \int a(t) dt \quad \rightarrow \quad \frac{\hat{\underline{a}}}{j\omega} e^{j\omega t}
    \]
\end{itemize}

\newpage
\section{Aufgaben}



\end{document}
